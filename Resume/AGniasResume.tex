% LaTeX resume using res.cls
\documentclass[line,resmargin,11pt]{res}
\usepackage[bottom=0.5in,top=0.5in, right=2in, left=0.5in]{geometry}
\usepackage{hyperref}
\urlstyle{same}

\begin{document}
\name{Andy Gnias}
\begin{resume}

\section{PROFILE}
Software engineer with extensive development and DevSecOps experience. Knowledge of multiple languages and frameworks, with a strong ability to think Pythonically. Skilled in automating DevSecOps processes, such as packaging, deployment, and end-to-end testing for various enterprise programs. Passionate about code coverage and extensive documentation when developing new code.
 
\section{EDUCATION}
Graduate Certificate in Computer Science \\
Temple University, Philadelphia, PA \\
Expected Graduation: May 2023

B.S. in Mechanical Engineering, June 2016 \\
Drexel University, Philadelphia, PA \\
GPA: 3.86

\section{COURSE- WORK}
	\textbf{Graduate Courses:} Programming Techniques (Data Structures and Algorithms) \\
	\textbf{Undergraduate Courses:} Computer Programming I and II (OOP in C++), Advanced Programming Techniques, Data Structures and Algorithms, Control Systems, Finite Element Methods, Computer Aided Design, Fluid Dynamics, Materials, Thermodynamics

\section{CERTIFI-CATIONS}
	\textbullet\ Machine Learning | Coursera Course, Stanford Online | May 2020 \\
	\textbullet\ Reinforcement Learning | Coursera Specialization, University of Alberta | Feb 2020 \\
	\textbullet\ Industrial Internet of Things | Lockheed Martin Pipeline Course | Oct 2018

\section{SOFTWARE}
\textbullet\ \textbf{Languages:} Python, C++, Java, JavaScript (Node.js), Perl, Bash, MATLAB  \\
\textbullet\ \textbf{Tools:} AWS, Docker, GitLab, Jenkins, Selenium \\
\textbullet\ \textbf{Engineering Tools:} ANSYS, Creo, MakerBot 3D printing, Minitab, LabVIEW, \LaTeX        
 
\section{EXPERIENCE}
\textbf{The Stratagem Group $|$ King of Prussia, PA} \\
{\sl Tech Lead - Cloud Development Program} \hfill September 2021 - Present
	\begin{itemize} \itemsep -2pt % reduce space between items
        \item Design and implement backend REST API services using Spring Boot in Java
        \item Implement unit tests using JUnit and Mockito
		\item Analyze existing projects for reusability on new program tasks
	\end{itemize}

{\sl Software Engineer Senior - Application Framework \& Infrastructure Program} \hfill September 2020 - October 2021
	\begin{itemize} \itemsep -2pt % reduce space between items
		\item Defined and implemented program-wide DevSecOps pipeline that can build, test, scan, and package C++, Java, and Python applications through GitLab
		\item Developed libraries in C++, Java, and Python to simplify API calls
		\item Deployed and tested applications in an OpenShift pipeline
		\item Served as scrum master for an agile development team
	\end{itemize}

{\sl Software Engineer Senior - Automated Testing Program} \hfill August 2019 - December 2020
	\begin{itemize} \itemsep -2pt % reduce space between items
		\item Created an AWS Lambda application to transfer TestRail project data  into a remote TestRail instance utilizing REST API calls
		\item Developed automated UI tests using Cucumber.js and Selenium to simulate user actions for over 40 different test scenarios
		\item Created automated test pipeline templates in Jenkins with "plug-and-play" usability
		\item Presented and demonstrated custom software tools at conferences to stakeholders
	\end{itemize}
	
{\sl Software Engineer Senior - Machine Learning Program} \hfill March 2020 - October 2020
	\begin{itemize} \itemsep -2pt % reduce space between items
		\item Determined useful attributes among datasets for accurate classification
		\item Created and improved applications interacting with data through Apache Kafka
		\item Hardened docker-compose development network to enhance functionality and stability
		\item Developed methods for hyperparameter optimization utilizing Optuna
	\end{itemize}

\textbf{Lockheed Martin, Space Systems Company $|$ King of Prussia, PA} \\
{\sl Software Engineer} \hfill April 2018 - August 2019 
	\begin{itemize}  \itemsep -2pt % reduce space between items
		\item Automated software installation, configuration, and testing as part of an agile development team
		\item Reduced source code by implementing a process to create environment-specific packages from a template
		\item Communicated with test users to help identify and resolve product defects
		\item Supported weekly product deployments to a factory level test environment
	\end{itemize}
	
{\sl Associate Software Engineer} \hfill September 2016 - April 2018
	\begin{itemize}  \itemsep -2pt % reduce space between items
		\item Maintained software archives in Nexus via RESTful API modules written in Perl
		\item Developed and improved End-to-end tests using Selenium and C\#
		\item Configured SSL/TLS certs for web services and applications
	\end{itemize}
				
\textbf{University City Science Center $|$ Philadelphia, PA} \\
{\sl Technical Investment Analyst} \hfill September 2014 - March 2015 
	\begin{itemize} \itemsep -2pt
		\item  Collaborated with local institutions to transition biomedical research into marketable products
		\item  Prepared and submitted grant applications through the National Institute of Health (NIH) and Small Business Innovation Research (SBIR) programs
	\end{itemize}
				
\textbf{Essential Medical $|$ Malvern, PA} \\
{\sl Product Development Co-op} \hfill September 2013 - March 2014
	\begin{itemize} \itemsep -2pt
		\item  Built and tested 500+ vascular closure devices for sealing capability and functionality
		\item  Wrote documentation, including Assembly Instructions and a Verification Test Plan
		\item  Conducted product experiments in an in vitro model and assisted in an in vivo study
	\end{itemize}

\section{LAB / PROJECT EXPERIENCE}
\textbf{Optical Diagnostics Lab, Drexel University $|$ Philadelphia, PA} \\
{\sl Hess Undergraduate Research Scholar} \hfill June 2015 - March 2016
	\begin{itemize}  \itemsep -2pt % reduce space between items
		\item  Wrote a Python script to monitor vibration on lab equipment using an accelerometer connected to a Raspberry Pi
		\item  Designed circuit board to read data from accelerometer and temperature monitor
	\end{itemize}

\textbf{Drexel University and Children's Hospital of Philadelphia $|$ Philadelphia, PA} \\
{\sl Senior Design Team Member} \hfill September 2015 - June 2016 \\
\begin{itemize} \itemsep -2pt
	\item  Optimized a laryngoscope blade to improve the success rate of pediatric intubation
	\item  Developed and read input from sensors to detect pressure points on device
	\item  Analyzed MRI scans of pediatric airway anatomy to determine optimal device curvature
\end{itemize}

\end{resume}
\end{document}
